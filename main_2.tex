\documentclass[12pt]{amsart}

%%%Titles and Authors%%%%%%%%%%%%%%%%%%%%%%
\title{Tensor stable moduli stacks and refined representations of quivers}

\author{Tarig Abdelgadir}
\address{The Abdus Salam International Centre for Theoretical Physics, 
Stada Costiera 11, 
Trieste 34151, 
Italy
}
\email{tabdelga@ictp.it}

\author{Daniel Chan}
\address{School of Mathematics and Statistics, 
UNSW Sydney, 
NSW 2052,	
Australia
}
\email{danielc@unsw.edu.au}

%-------------------------Packages-------------------------
\usepackage{mymacros,amssymb}
\usepackage{fullpage,tikz}
\usepackage{tikz-cd}
\usepackage{pst-node}
\usetikzlibrary{graphs,angles,quotes,positioning,calc,arrows.meta}
\usepackage[all]{xy}

%-----------------------Macros-------------------------------
\newcommand{\Wt}{\textup{Wt}}

%-------------------------Text Starts----------------------

\begin{document}
\maketitle

\section{Introduction}

Moduli spaces are a fruitful way to study a $k$-linear abelian category {\sf C}.
For example, for $G \subset \SL(2,k)$ a finite group we may take {\sf C} to be the category of $G$-equivariant $k[x,y]$-modules or take take {\sf C} to be a noncommutative deformation of coherent sheaves on a projective plane. 
In the first case, the moduli space of objects in {\sf C} with natural discrete invariants recovers the minimal resolution of $\bA^2/G$.
In the later, the moduli space of point modules, \`a la Artin-Tate-van den Bergh \cite{ATV}, gives an elliptic curve.
This curve along with its universal line bundles gives us a presentation of the {\sf C} as the category of graded modules on a twisted homogeneous coordinate ring.

The above examples may tempt one to think that the only interesting moduli space to study is the moduli stack $\bM$ of isomorphism classes of objects in {\sf C}.
Sometimes, however, that is not the case.
In many settings, the categories in question lend themselves better to Deligne-Mumford (DM) stacks, none more so than $\text{\sf C}=\Coh(\bX)$ for a DM stack $\bX$.
In this case, as in other such {\sf C}, there are at least two obstructions to expressing $\bX$ as tautological moduli problem on $\Coh(\bX)$. 
Firstly, the na\"ive analogue of Gabriel-Rosenberg theorem of reconstructing schemes from their abelian categories of quasi-coherent sheaves fails for DM stacks.
The other obstruction is the fact that objects in abelian categories have connected automorphism groups.

Lurie's Tannakian duality for geometric stacks \cite{Lurie} provides the inspiration to get around these issues: Theorem 5.11 in \cite{Lurie} states that one may recover $\bX$ from its monoidal abelian category $(\Coh(\bX), \otimes)$.
Therefore the key to posing a moduli problem that would $\bX$ from {\sf C} is to enhance the data of an object in $\bX$ with data from the monoidal structure on {\sf C}.
This will be the strategy we adopt in this article.
The second issue, the one regarding the connectedness of the automorphism, happens to be a result of the first and disappears once one has made the correct monoidal adjustments as is suggested by the classical Tannakian duality for finite groups.

In this article, we begin attempting to address the need for a {\em tautological moduli problem} on a DM stack $\bX$.
Now when $\bX=X$ is a separated scheme, there is an obvious choice for this tautological moduli problem. Take $\Delta \subset X \times X$ to be the diagonal, then $\cO_{\Delta}$ is the universal skyscraper sheaf on $X$. 
For stacks, it is not clear what the correct notion of a skyscraper sheaf is since they can ``fractionate'' in the language of physics. 
In Section~\ref{sec:sky}, we address this problem and define the moduli stack $\bM$ of skyscraper sheaves.
The way we incorporate the monoidal structure of $\Coh(\bX)$ into $\bM$ is inspired by ideas in Chan-Lerner \cite{CL}.
We consider line bundles $\cL_1,\ldots, \cL_s \in \Pic \bX$ and the corresponding rationally defined self-maps $\cL_i \otimes_{\bX} (?) \colon \bM \dashrightarrow \bM$. 
The {\em tensor stable moduli stack} $\bM^{\cL_1,\ldots,\cL_s}$ is defined in Definition~\ref{defn:doubletensor} as a mild modification of the simultaneous fixed point stack. 
One of our main results is the following rather imprecise re-statement of Theorem~\ref{thm:tautological}.
\begin{theorem}
Let $\bX$ be a separated quasi-projective stack and suppose that $\cL_1\oplus \ldots \oplus \cL_s$ is faithful. Then $\bM^{\cL_1,\ldots,\cL_s} \simeq \bX$.
\end{theorem}
\noindent Unfortunately, the hypothesis on $\cL_1\oplus \ldots \oplus\cL_s$ implies that the stabiliser groups of $\bX$ are all abelian. 
However, it applies in particular, to the weighted (a.k.a. G-L) projective spaces of \cite{HIMO}, an example we look at in Section~\ref{sec:HIMO}.

Another setting in which stacks are natural is that of Ringel's canonical algebras which by Geigle-Lenzing \cite{GL} are derived equivalent to to weighted projective lines.
Indeed, only recently, Abdelgadir-Ueda \cite{AU} and \cite{CL} introduced new moduli stacks to exhibit the Geigle-Lenzing derived equivalences as Fourier-Mukai transforms induced by universal modules on appropriate moduli stacks.
The ideas therein are driving forces behind the work done here.
What is striking, however, is how differently \cite{AU} and \cite{CL} incorporate the monoidal structure needed for stack recovery. 
The latter introduces the {\em Serre stable moduli stack} which essentially is the fixed point stack of the rational self map $S\colon \bM \dashrightarrow \bM$ where $S$ is induced by a cohomological shift of the Serre functor (see Section~\ref{sc:equaliserstack}). 
The former involves moduli of {\em refined representations} where the data of an $A$-module is enriched by {\em refinement data} (see Definition~\ref{def:refined}).
Addressing this apparent discrepancy is one of the goals of this work.

Furthermore, both methods have their limitations and in no way claim to provide a general framework.
For example, the Serre stable moduli stack fails when the stack has non-cyclic stabilisers and is of dimension greater than 1. 
This suggests that there are lots of interesting moduli stacks out there waiting to be discovered.
One major goal of this paper is to construct such moduli stacks.

In a similar fashion to \cite{GL}, let $\bX$ be a DM stack that posses a tilting bundle $\cT$ that is a direct sum of line bundles.
For the endomorphism algebra $A = \End_{\bX} \cT$, we show in Section~\ref{sc:tensor_stable} how to construct the analogous tensor stable moduli stack $\bM^{L_1,\ldots,L_s}$ of $A$-modules. 
Theorem~\ref{thm:stackfromquiver}, under mild hypotheses, gives a modular realisation of the derived equivalence induced by $\cT$.
This follows the ideas leading to the Serre-stable moduli space in \cite{CL}.
In the case where $\bX$ is a scheme, the result here is similar to that in \cite{MR2421120}; it is, however, important to note that the theorem in this paper uses the full force and naturality of moduli theory and does not rely on smoothness or pointwise arguments.

Staying in the setting where $\bX$ posses such a tilting bundle, we take the quiver route, i.e.\ $A \cong kQ/I$ for some quiver $Q$.
Furthermore, the moduli space $\bM$ is then a moduli space of quiver representations.
In Section~\ref{sec:refined}, we enhance families of quiver representations with {\em refinement data} encoding the required monoidal structure.
We go on to construct moduli spaces $\bM_\text{ref}$ parametrising them and eventually get the following result.
\begin{theorem}  \label{thm:samemoduli}
Let $\cT$ be a tilting bundle on $\bX$ whose components are line bundles. 
Then there is an open subset $\bM_{\textup{ref}}'$ of $\bM_{\textup{ref}}$ is isomorphic $\bM^{L_1,\ldots,L_s}$ and hence $\bX$.
\end{theorem}

On the face of it the moduli of refined representations seems a long way away from that used in defining the tensor-stable moduli space.
In $\bM_\text{ref}$ an $S$-point is supplemented with the refinement data $g$ while in $\bM^{L_1,\ldots,L_s}$ it is enriched by different {\em tensor stability} data which we will denote by $\psi$.
However, the proof of Theorem~\ref{thm:samemoduli} directly relating these two bits of data and showing that their corresponding moduli problems give (essentially) the same answer.
This unites the approaches developed in \cite{AU} and \cite{CL}.

One advantage the moduli of refined representations has over the tensor stable moduli stack, is that the former is naturally a quotient stack and can be studied via GIT.
In particular, we expect the open subset $\bM_\text{ref}'$ referred to in Theorem~\ref{thm:samemoduli} to be defined by a GIT stability parameter $\theta$.
Furthermore, in especially nice settings this GIT construction gives a more direct proof of $\bX \simeq \bM_\text{ref}'$ than the one through the identification with $\bM^{L_1,\ldots,L_s}$.
These aspects of the stack $\bM_\text{ref}$ are discussed in Section~\ref{sc:GIT} while the special case of G-L projective spaces is addressed in Section~\ref{sec:HIMO}.

We further note here that the theory developed for refined representations of quivers may be developed for endomorphism algebras for any collection of effective line bundles on $\bX$ not just those coming from tilting bundles.
Analogues of multilinear series introduced by Craw-Smith \cite{Craw-Smith} can be developed in the setting of the projective stacks in question here following on from \cite{Abd} where toric stacks are discussed. 


\section{Background: quasi-projective stacks}

In this paper, we will mainly be dealing with quasi-projective stacks $\bX$ as defined by Kresch \cite{Kr}. We record here his definition as well as some basic facts about such stacks.

Let $\bX$ be a Deligne-Mumford stack of finite type. We will also assume that $\bX$ is separated, or more generally, has {\em finite inertia}, which just means that the inertia stack $$\mathbb{I}(\bX) :=\bX \times_{\Delta,\bX \times \bX , \Delta} \bX$$ 
is finite over $\bX$. We know from \cite{KeM} that there is a coarse moduli space $X$ and we let $c\colon \bX \to X$ denote that canonical quotient morphism. Furthermore, \'etale locally on $X$, $\bX$ is isomorphic to a quotient stack of the form $[U/G]$ where $G$ is a finite group.

Our characteristic 0 assumption ensures that $\bX$ is a tame stack in the sense of \cite{AOV} so in particular, in the terminology of \cite[Definition~3.1]{Alp}, the morphism $c \colon \bX \to X$ is {\em cohomologically affine} in the sense that $c$ is quasi-compact and $c_* \colon \Qcoh(\bX) \to \Qcoh (X)$ is exact.  From \cite[Proposition~3.10]{Alp}, we know that cohomologically affine morphisms are stable under compositions and base change if the bases are Deligne-Mumford stacks. 

We recall Alper's projection formula \cite[Proposition~4.5]{Alp}.
\begin{proposition}   \label{prop:projection}
Let $f \colon \bY \to Y$ be a cohomologically affine morphism of Artin stacks where $Y$ is an algebraic space. Then the natural morphism below is an isomorphism for quasi-coherent sheaves $\cF, \cF'$ on $\bY, Y$, respectively.
$$ f_* \cF \otimes_{Y} \cF' \to f_* (\cF \otimes_{\bY} f^* \cF').$$
\end{proposition}

Following \cite{OS} we define
\begin{definition}
A coherent locally free sheaf $\cG$ on $\bX$ is a {\em generating sheaf} if the natural morphism
$$ c^*(c_*\cHom_{\bX}(\cG, \cF)) \otimes_{\bX} \cF \to \cF$$
is surjective for every quasi-coherent sheaf $\cF$ on $\bX$. 
\end{definition}
By \cite[Theorem~5.2]{OS}, this condition can be checked geometrically pointwise as follows. Given any geometric point $\xi \colon \operatorname{Spec} k \to \bX$, we consider the (geometric) stabiliser group $G_{\xi} := \operatorname{Spec} k \times_{\xi, \bX} \mathbb{I}(\bX)$ which is a finite group. Then a locally free sheaf $\cG$ generates if and only if the $G_{\xi}$-module $\xi^* \cG$ generates $\textup{Mod-} kG_{\xi}$ for every geometric point $\xi$. 


The importance of this concept for us, is that it allows us to relate the theory of stacks to non-commutative algebraic geometry. 
\begin{proposition} \label{prop:Morita}
Fix a generating sheaf $\cG \in \Qcoh(\bX)$ and define $\cA := c_* \cEnd (\cG)$ then 
\begin{align*}
\Phi \colon \Qcoh (\bX) &\longrightarrow \textup{Mod-}\cA \\
 \cF & \longmapsto c_* \cHom(\cG, \cF)
\end{align*}
is an equivalence of categories.
\end{proposition}

\begin{proof}
First note that $\Phi$ is exact and that $\cG$ is locally free. Moreover, $\Phi$ admits a left adjoint:
\begin{align*}
\Psi \colon \textup{Mod-}\cA &\longrightarrow \Qcoh (\bX) \\
 M & \longmapsto c^*M \otimes_{c^* \cA} \, \cG.
\end{align*}
We begin by showing that the composite $\Phi \circ \Psi$ is isomorphic to the identity. 
Since $\cG$ is a generating sheaf, we may present a general $\cF \in \Qcoh(\bX)$ as follows: $$c^* \cV_1 \otimes_\bX \cG \longrightarrow c^* \cV_2 \otimes_\bX \cG \longrightarrow \cF$$ where $\cV_1, \cV_2 \in \Qcoh(X)$ are locally free. 
Hence it suffices to show that the adjuction morphism $\Phi \circ \Psi (c^* \cV \otimes_\bX \cG) \longrightarrow c^* \cV \otimes_\bX \cG$ is in fact an isomorphism.
This then follows from the following chain of isomorphisms:
\begin{align*}
\Phi \circ \Psi (c^* \cV \otimes_\bX \cG) &\cong c^*(c_* \cHom(\cG, c^* \cV \otimes_\bX \cG) \otimes_{c^* \cA} \, \cG \\
& \cong c^*(c_* (c^* \cV \otimes_\bX \cEnd(\cG))) \otimes_{c^* \cA} \, \cG \\
& \cong c^* (\cV \otimes_X \cA) \otimes_{c^* \cA} \, \cG \\
& \cong c^*\cV \otimes_\bX \cG.
\end{align*}

It remains to show that $\Psi \circ \Phi$ is isomorphic to the identity.
For $M \in \textup{Mod-}\cA$, one may show that the adjuction morphism $M \rightarrow \Psi \circ \Phi(M)$ is an isomorphism it suffices by checking it locally on the coarse moduli space. 
Locally over the coarse moduli space we may freely present $M \in \textup{Mod-}\cA$ as follows: $$\cA^m \longrightarrow \cA^n \longrightarrow M \longrightarrow 0.$$ 
The result then follows from the observation that $A \longrightarrow \Psi \circ \Phi (\cA)$ is an isomorphism.
\end{proof}

\begin{remark}  \label{rem:Morita}
Many results about quasi-coherent sheaves on stacks can be reduced corresponding results on schemes using Proposition~\ref{prop:Morita}. This seems to be a relatively easy way to check results concerning stacks for those less familiar with the theory. We will use this a number of times.
\end{remark}

\begin{definition}
We say $\bX$ is {\em (quasi-)projective} if it has a generating sheaf and the coarse moduli space $X$ is a (quasi-)projective scheme.
\end{definition}

Examples of projective stacks include weighted projective lines as introduced by Geigle-Lenzing \cite{GL} and the class of stacks introduced in Herschend-Iyama-Minamoto-Opperman \cite{HIMO}.
The authors of \cite{HIMO} chose to name them Geigle-Lenzing projective spaces in honour of \cite{GL}.

\section{Skyscraper sheaves on stacks}
\label{sec:sky}

In this section, we introduce the notion of a skyscraper sheaves on a quasi-projective stack $\bX$ and their moduli. By definition, the coarse moduli space $X$ is a quasi-projective scheme and there exists a generating sheaf $\cG$ which is far from being unique. Let $c \colon \bX \rightarrow X$ be the canonical morphism to the coarse moduli scheme and suppose there is a fixed decomposition $\cG = \oplus_i \cG_i$. Let $\cA := c_* \cEnd_{\bX} \cG$ which is a finite sheaf of algebras on $X$.

Let $\tilde{\bM}_{\Coh}$ be the moduli stack of coherent sheaves on $\bX$. Recall that $\Coh(\bX)$ is a $k$-linear category, so the inertia groups of every object contain a copy of $\mathbb{G}_m$. We remove this common copy of $\mathbb{G}_m$ by {\em rigidification} as defined for example in \cite[Section~5]{ACV} (see also \cite[Section~2.3]{CL} for a gentle description). Let $\bM_{\Coh}$ denote the resulting rigidified moduli stack of coherent sheaves on $\bX$. It can be described as the stackification of the following pre-stack $\bM^{pre}$. Given a test scheme $T$, the objects of $\bM^{pre}(T)$ are those of $\tilde{\bM}(T)$. Given objects $\cM, \cN \in \bM^{pre}(T)$, the isomorphisms from $\cM$ to $\cN$ consist of equivalence classes of isomorphisms $\theta \colon \cM \to \cM' \otimes_T \cN$ where $\cN$ is a line bundle on $T$ and $\theta' \colon \cM \to \cM' \otimes_T \cN'$ is {\em equivalent} to $\theta$ if there is some isomorphism $l \colon \cN \to \cN'$ with $\theta' = (\id \otimes l) \theta$. The pullback pseudo-functor is that induced from $\tilde{\bM}_{\Coh}$. 

We will always rigidify our moduli stacks in this way so in general will omit the adjective ``rigidified''. We next define the (rigidified) moduli stack $\bM_{\textup{Fin}}$ of finite length sheaves on $\bX$. The objects over a test scheme $T$ consist of coherent sheaves $\cF \in \Qcoh(\bX \times T)$ which are flat over $T$ and such that the support $Z \subseteq X \times T$ of $\Phi_{\cG}(\cF):=(c\times \id_T)_*\cHom_{\bX}(\cG, \cF)$ is finite over $T$. Since flatness is a property of a Grothendieck category (see \cite{AZ01}), this is equivalent by Proposition~\ref{prop:Morita}, to a flat family of coherent $\cA$-modules with finite support. The morphisms in $\bM_{\textup{Fin}}$ are defined to be the same as for the moduli stack of coherent sheaves on $\bX$. We note that the definition is independent of the choice of generator, since Morita equivalences of finite sheaves of algebras over $X$ preserve support on $X$.

\begin{proposition}  \label{prop:MFinisartin}
The moduli stack $\bM_\textup{Fin}$ is an Artin stack.
\end{proposition}
\begin{proof}
Suppose first that $\bX$ and hence $X$ are projective so that we may speak of Hilbert polynomials of coherent $\cA$-modules with respect to some fixed choice of an ample line bundle. We may think of $\bM_{\Coh}$ as the moduli stack of coherent $\cA$-modules and then $\bM_{\textup{Fin}}$ consists of those components of $\bM_{\Coh}$ whose Hilbert polynomials are bounded. 

For quasi-projective $X$, we may pick a projective closure $\bar{X}$ of $X$ and extend $\cA$ to a coherent sheaf of algebras $\bar{\cA}$ on $\bar{X}$. It suffices then to show that $\bM := \bM_{\textup{Fin}}$ is an open substack of the moduli stack $\bar{\bM}$ of finite length $\bar{\cA}$-modules. 
Let $\cF$ be a flat family of $\cA$-modules over $T$ whose support $Z \subseteq X \times T$ is finite $i\colon X \times T \to \bar{X} \times T$ and $j \colon Z \to \bar{X} \times T$ be the natural embeddings. Then $i_* \cF = j_* \cF$ defines a flat family of $\bar{\cA}$-modules of finite length. This exhibits $\bM$ as a substack of $\bar{\bM}$ and it only remains to observe that the condition of being in $\bM$ is open. 
\end{proof}

To obtain analogues of the notion of skyscraper sheaves, we need some discrete invariants. The following helps us define such invariants. 
\begin{proposition}  \label{prop:cstarflat}
Let $\cF \in \Coh(\bX \times T)$ be a family of sheaves which is flat over $T$ and $\pi_1 \colon \bX \times T \to \bX, \pi_2 \colon \bX \times T \to T$ be the projection maps. Then for any coherent locally free sheaf $\cV \in \Coh(\bX)$, the sheaf $(c\times \id_T)_*\cHom_{\bX \times T}(\pi_1^*\cV, \cF)$ is flat over $T$. In particular, if $\cF$ is a flat family of finite length sheaves, then $\pi_{2*}\cHom_{\bX \times T}(\pi_1^*\cV, \cF)$ is a locally free sheaf on $T$. 
\end{proposition}
\begin{proof}
Since $\cHom_{\bX \times T}(\pi^*\cV, \cF)$ is also flat over $T$, it suffices, for the first assertion, to show that $(c \times \id_T)_* \cF$ is flat over $T$. To this end, consider an injection of quasi-coherent sheaves $M' \hookrightarrow M$ on $T$. Flatness of $\cF$ means that we have an injection $\cF \otimes_{\bX \times T} \pi_2^* M' \hookrightarrow \cF \otimes_{\bX \times T} \pi_2^* M$. The projection formula in Proposition~\ref{prop:projection} and the fact that $c \times \id_T$ is cohomologically affine now shows that the natural map
$$ (c \times \id_T)_* \cF \otimes_{X \times T} \pi_2^*M' \hookrightarrow (c \times \id_T)_* \cF \otimes_{X \times T} \pi_2^*M$$
is injective. Hence $(c \times \id_T)_* \cF$ is indeed flat over $T$. The second assertion now follows from \cite[Proposition~9.2(d)]{Hart}. 
\end{proof}

The proposition allows us to make the following definition. 
\begin{definition}
Let $\cF\in \cM_{\textup{Fin}}(T)$ be a flat family of finite length sheaves on $\bX$ over $T$ and $\cV$ be a coherent locally free sheaf on $\bX$. The {\em $\cV$-rank of $\cF$} is defined to be 
$$\cV\!-\!\rank \cF := \rank_T \pi_{2*} \cHom_{\bX \times T}(\pi_1^* \cV, \cF).$$
We say that $\cF$ has {\em skyscraper $\cV$-rank} if $\cV\!-\!\rank \cF = \rank \cV$. We define the {\em moduli stack $\bM_{\textup{Sky}}$ (resp. $\bM_{\cG-\textup{Sky}}$) of skyscraper sheaves (resp. relative to $\cG = \oplus \cG_i$)}, to be the substack of $\bM_{\textup{Fin}}$ consisting of $\cF \in \bM_{\textup{Fin}}$ with skyscraper $\cV$-rank for every coherent locally free sheaf $\cV$ on $\bX$ (resp. for all $\cV = \cG_i$). 
\end{definition}

The $\cV$-rank is an important discrete invariant we can use to decompose the moduli stack of finite length sheaves. These invariants are of course, not all independent and it is useful to know the relationships between them. For example, we have

\begin{proposition}  \label{prop:Vrankcoarse}
Let $\cF \in \bM_{\textup{Fin}}(T)$. Suppose $\cV$ is a coherent locally free sheaf  on $\bX$ and $\cW$ is a coherent locally free sheaf on $X$ of rank $r$. Then
$$ (c^* \cW \otimes_{\bX} \cV)\!-\!\rank \cF = r (\cV\!-\!\rank \cF).$$
\end{proposition}
\begin{proof}
The projection formula (Proposition~\ref{prop:projection}) ensures that 
$$ \pi_{2*} \cHom_{\bX \times T}(\pi_1^*c^*\cW \otimes_{\bX \times T} \pi_1^* \cV, \cF) \simeq 
\pi_{2*} \left(\pi_1^*\cW \otimes_{X \times T} (c\times \id)_* \cHom_{\bX \times T}(\pi_1^* \cV, \cF)\right)$$
where we have abused notation by letting $\pi_1,\pi_2$ denote projections from both $\bX \times T$ and $X \times T$. Now the support of $\cHom :=(c\times \id)_* \cHom_{\bX \times T}(\pi_1^* \cV, \cF)$ is finite, so locally on $T$ we have 
$$ \pi_1^* \cW \otimes_{X \times T} \cHom \simeq \cHom^{\oplus r}.$$
\end{proof}

Recall that the diagonal map $\Delta \colon \bX \to \bX \times \bX$ is a representable morphism which is not a monomorphism unless $\bX$ is an algebraic space. If $\bX$ is a separated DM-stack, then by definition, $\Delta$ is finite and in particular, affine. We may consider $\Delta_* \cO_{\bX}$ as a family of coherent sheaves on $\bX$ over $\bX$. To be explicit, we will consider the second factor in $\bX \times \bX$, the base space for the family. 

\begin{notation} \label{notn:bimodtensor}
Given algebraic stacks $\bX, \bY$, and quasi-coherent sheaves $\cF_X \in \Qcoh(\bX), \cF \in \Qcoh(\bX \times \bY), \cF_Y \in \Qcoh(\bY)$ we let 
$$ \cF_X \otimes_{\bX} \cF \otimes_{\bY} \cF_Y :=
\pi_X^* \cF_X \otimes_{\bX\times \bY} \cF \otimes_{\bX \times\bY} \pi_Y^*\cF_Y $$
where $\pi_X,\pi_Y$ are the projection maps. 
\end{notation}

To get a feel for $\Delta_* \cO_{\bX}$, we consider the special case $\bX= [U/G]$ where $U$ is a quasi-projective scheme and $G$ is a finite group acting on $U$. If we wish to view coherent sheaves on $\bX$ as $G$-equivariant sheaves on $\bX$, then we should pull back $\Delta_* \cO_{\bX}$ via $\pi \times \id_{\bX} \colon U \times \bX \to \bX \times \bX$ where $\pi \colon U \to \bX$ is the canonical quotient map, and remember the $G$-action. We will study this family over $\bX$ by pulling back to a family over $U$. Hence consider the cartesian diagram
$$\begin{CD}
G \times U @>>> \bX \\
@V{\delta}VV @VV{\Delta}V \\
U \times U @>{\pi \times \pi}>> \bX \times \bX 
\end{CD}.$$
Here $\delta = (\alpha, pr_2)$ where $\alpha \colon G \times U \to U$ is the action and $pr_2$ is the projection map. Thus $\Delta_*\cO_{\bX}$ when pulled 
back to a family on $U$ is given by the $G$-equivariant sheaf $\delta_* \cO_{G \times U}$. It is useful to view this as the skew group ring $G \# \cO_U$ where left multiplication by $G$ and $\cO_U$ give the structure of a $G$-equivariant sheaf, and right multiplication by $\cO_{U}$ determines the geometry of the family.  Note that the support of $G \# \cO_U$ as a sheaf on $U \times U$ is $Z = \cup_{g \in G} (g,\id)(U)$ which is finite over $U$. Also, if we pick a geometric point $\Spec K$ of the base $U$, then the corresponding $G$-equivariant sheaf is $G \# \cO_U \otimes_U K \simeq G \# K$, the regular representation of $G$ over $K$.
In other words, $\Delta_* \cO_\bX$ is analogous to the universal family on the corresponding $G$-Hilbert scheme.


\begin{proposition} \label{prop:universalsky}
Suppose that $\bX$ is a separated quasi-projective stack. Then $\Delta_* \cO_{\bX}$ is a flat family of skyscraper sheaves over $\bX$. 
\end{proposition}
\begin{proof}
Now $\Delta$ is representable so the projection formula shows that for locally free $\cV_1,\cV_2 \in \Coh(\bX)$ we have
\begin{equation} \cV_1 \otimes_{\bX} \Delta_* \cO_{\bX} \otimes_{\bX} \cV_2 \simeq \Delta_*(\cV_1 \otimes_{\bX} \cV_2) .
\label{eq:projsky}
\end{equation}
Also, our assumption that $\bX$ is separated ensures now that $\Delta_*$ is exact so the same is true for $\Delta_* \cO_{\bX} \otimes_{\bX} -$ on calculating Tor. It follows that $\Delta_* \cO_{\bX}$ is indeed a flat family of coherent sheaves. 

We check now that the support $Z$ of $\Delta_* \cO_{\bX}$ is finite over $\bX$. Now $\bX$ is \'etale locally  a quotient stack of the form $[U/G]$ where $U$ is a scheme and $G$ is a finite group, so this follows from the local computation above. 

Finally, (\ref{eq:projsky}) shows that 
$$ \pi_{2*} (\cV_1 \otimes_{\bX} \Delta_* \cO_{\bX})
\simeq \pi_{2*} \Delta_*(\cV_1) = \cV_1
$$
so the discrete invariants are precisely those of a skyscraper sheaf. 
\end{proof}

\section{Tensor stable moduli stacks} \label{sc:equaliserstack}

Let $\bM$ be an Artin stack and $f \colon \bM \dashrightarrow \bM$ be a partially defined morphism, that is, there is an inclusion of locally closed substacks $\iota \colon \bM' \to \bM$ and a morphism $f\colon \bM' \to \bM$. Consider the graph morphism $\Gamma = \Gamma_{f} \colon \bM' \xrightarrow{\Delta} \bM' \times \bM' \xrightarrow{\iota \times f} \bM \times \bM$. We define the {\em fixed point stack of $f$} to be the fibre product stack
$$ \bM^f = \bM \times_{\Delta,\bM \times \bM, \Gamma} \bM'.$$
The definition depends of course on the domains of definition $\bM'$ which is suppressed from the notation, but like the situation with rational maps in algebraic geometry, there is often a clear ``maximal'' choice. Since stacks are themselves categories, the notion of fixed point stacks exhibits is somewhat subtle ``higher'' categorical phenomena. For example, the fixed point stack of the identity morphism is actually the inertia stack, which is not necessarily the original stack. 

The key cases for us are where $\bM, \bM'$ are moduli stacks on some Grothendieck categories and the partially defined maps are induced by functors. For example, let $\bX$ be a separated quasi-projective stack, say with generator $\cG = \oplus \cG_i$ and $\bM =  \bM_{\cG-\textup{Sky}}, \bM' = \bM_{\textup{Fin}}$. Then tensoring by a rank $r$ vector bundle $\cV$ on $\bX$, is a functor from $\Coh(\bX) \to \Coh(\bX)$ which induces a morphism of stacks $\bM \to \bM'$. Suppose now that $\cL$ is a line bundle so tensoring by $\cL$ induces a partially defined maps $\bM \dashto \bM$. We let $\bM^{\cL}$ be the fixed point stack of $\bM$ with respect to the self-map induced by tensoring by $\cL$. Suppose now we are given two line bundles $\cL_1,\cL_2$. Now there is a canonical natural isomorphism $\cL_1 \otimes (\cL_2 \otimes -) \simeq \cL_2 \otimes (\cL_1 \otimes -)$, so $\cL_2 \otimes (-)$ induces a partially defined map on $\bM^{\cL_1}$ and we may define iteratively $(\bM^{\cL_1})^{\cL_2}$. 

\begin{proposition}  \label{prop:doublestable}
Given a test scheme $T$, an object of $(\bM^{\cL_1})^{\cL_2}(T)$ consists of a flat family $\cM \in \bM(T)$, isomorphisms $\theta_1 \colon \cL_1 \otimes_{\bX} \cM \simeq \cM \otimes_T \cN_1, \ \ \theta_2 \colon \cL_2 \otimes_{\bX} \cM \simeq \cM \otimes_T \cN_2$ where $\cN_1, \cN_2$ are line bundles on $T$ and the $\theta_i$ are defined up to scalar only, such that the following diagram commutes up to scalar
\begin{equation} \label{eq:doublestable}
\begin{CD}
\cL_1 \otimes_{\bX} \cL_2 \otimes_{\bX} \cM @>>> \cL_2 \otimes_{\bX} \cL_1 \otimes_{\bX} \cM @>{1 \otimes \theta_1}>> \cL_2 \otimes_{\bX} \cM \otimes_T \cN_1 \\
@V{1 \otimes \theta_2}VV @. @VV{\theta_2 \otimes 1}V \\
\cL_1 \otimes_{\bX} \cM \otimes_T \cN_2 @>{\theta_1 \otimes 1}>> \cM \otimes_T \cN_1 \otimes_T \cN_2 @>>> 
\cM \otimes_T \cN_2 \otimes_T \cN_1
\end{CD}.
\end{equation}
\end{proposition}
\begin{proof}
We begin by considering an object of $\bM^{\cL_1}(T)$. By definition of the product of stacks, this consists of the data of a pair $(\cM, \cM') \in \bM(T) \times\bM(T)$ and an isomorphism $\alpha \colon (\cM, \cM) \simeq (\cM', \cL_1 \otimes \cM')$ in $\bM(T) \times \bM(T)$. Since we are working by default with rigidified moduli stacks, this reduces to the data $(\cM, \theta_1)$ where $\cM \in \bM(T)$ and $\theta_1 \colon \cL_1 \otimes_{\bX} \cM \simeq \cM \otimes_T \cN_1$ is an isomorphism for some line bundle $\cN_1$ on $T$. This isomorphism is defined only up to scalar. Now tensoring by $\cL_2$ induces a partially defined automorphism of $\bM^{\cL_1}$ which sends $(\cM,\theta_1)$ to the pair 
$$(\cL_2 \otimes_{\bX} \cM, \cL_1 \otimes_{\bX} \cL_2 \otimes_{\bX} \cM \simeq \cL_2 \otimes_{\bX} \cL_1 \otimes_{\bX} \cM \stackrel{1 \otimes \theta_1}{\simeq} \cL_2 \otimes_{\bX} \cM \otimes_T \cN_1).$$
Note that we have used the natural isomorphism $\cL_1 \otimes (\cL_2 \otimes -) \simeq \cL_2 \otimes (\cL_1 \otimes -)$ here. An object of $(\bM^{\cL_1})^{\cL_2}(T)$ consists of an isomorphism between these two pairs. This is given by $\theta_2 \colon \cL_2 \otimes_{\bX} \cM \simeq \cM \otimes_T \cN_2$ such that the diagram~(\ref{eq:doublestable}) commutes up to scalar.
\end{proof}
This description of $(\bM^{\cL_1})^{\cL_2}$ makes clear the symmetry between $\cL_1, \cL_2$ so it does not matter which order we perform the fixed point stacks. Another interesting point is that given the object of $(\bM^{\cL_1})^{\cL_2}(T)$ defined by the data $(\cM, \theta_1, \theta_2)$ above, there is a well-defined scalar $\lambda \in \cO_T^{\times}$ such that in diagram~(\ref{eq:doublestable}) we have $(\theta_2 \otimes 1) (1 \otimes \theta_1) = \lambda (\theta_1 \otimes 1) (1 \otimes \theta_2)$. Indeed changing either $\theta_1$ or 
$\theta_2$ by a scalar does not affect $\lambda$. The formation of this scalar is compatible with pullback in the stack so there is a well-defined morphism of stacks $\nu \colon (\bM^{\cL_1})^{\cL_2} \rightarrow \bG_m$. 

\begin{definition}  \label{defn:doubletensor}
We let $\bM^{\cL_1,\cL_2}$ be the fibre product stack $(\bM^{\cL_1})^{\cL_2} \times_{\bG_m} 1$ where the map $1 \hookrightarrow \bG_m$ is the inclusion of the unit map. The data of an object over $T$ consists of $(\cM, \theta_1, \theta_2)$ as in Proposition~\ref{prop:doublestable} but now where Diagram~(\ref{eq:doublestable}) commutes on the nose. Similarly given line bundles $\cL_1,\ldots,\cL_s$ on $\bX$, we define $\bM^{\cL_1,\ldots,\cL_s}$ and call it the {\em tensor stable moduli stack of skyscraper sheaves with respect to line bundles $\cL_1, \ldots, \cL_s$ (and generator $\cG$)}. We refer to the isomorphisms $\theta_i$ as {\em tensor stability data}.
\end{definition}



\section{Tautological moduli problem}

In this section, we exhibit a tautological moduli problem in the special case where inertia groups are all abelian.

Let $\cL_1, \ldots, \cL_s$ be line bundles on a quasi-projective stack $\bX$. Let $c \colon \bX \to X$ be the canonical morphism to the coarse moduli scheme. Suppose that the geometric stabliser groups act faithfully on $\oplus \cL_i$, in which case we say that $\oplus \cL_i$ is a faithful bundle. In this case, the geometric stabilisers are abelian, and conversely, given such a quasi-projective stack with abelian geometric stabilisers, there exists \'etale locally on $X$, a faithful direct sum of line bundles. Burnside's theorem ensures that $\bX$ has a generating sheaf $\cG$ which is  a direct sum of line bundles constructed by tensoring the $\cL_i$ together. We wish to study the tensor stable moduli stack of skyscraper sheaves $\bM^{\cL_1,\ldots,\cL_s}$ with respect to $\cL_1, \ldots, \cL_s$ and some appropriate generator $\cG$. 

The following is standard in stack theory. Recall that if $\cB$ is a $\bZ^s$-graded sheaf of algebras on $X$, then there is an action of $(\bZ^s)^{\vee} = \bG_m^s$ on $P = \underline{\Spec}_X \cB$. 
\begin{proposition}  \label{prop:XisPmodG}
With the above hypotheses, there exists a $\bZ^s$-graded sheaf of algebras $\cB = \oplus \cB_{\chi_1,\ldots,\chi_s}$ on $X$ 
%with each $\cB_{i_1,\ldots,i_s}$ coherent and 
such that $\bX \simeq [P/\bG_m^s]$ where $P = \underline{\Spec}_X \cB$. 
\end{proposition}
\begin{proof}
Let 
$$P = \underline{\Spec}_{\bX} \bigoplus_{(\chi_1,\ldots,\chi_s)\in \bZ^s} \cL_1^{\otimes \chi_1} \otimes_{\bX} \otimes \cdots \otimes_{\bX} \cL_s^{\otimes \chi_s}.$$
Our assumption that $\oplus \cL_i$ is faithful means that this is an algebraic space. By construction, we have $\bX \simeq [P/\bG_m^s]$. Now $P \to \bX$ is affine and hence, cohomologically affine, whilst $c \colon \bX \to X$ is cohomologically affine so the same is true of the composite $f \colon P \to X$. The algebraic space version of Serre's criterion for affineness \cite[Proposition~3.3]{Alp} ensures that $f$ is actually affine so $P = \underline{\Spec}_X \cB$ for some sheaf of algebras $\cB$. Furthermore, the action of $\bG_m^s$ on $P$ induces a $\bZ^s$-graded structure on $\cB$. 
\end{proof}
The proposition allows us to identify $\Qcoh(\bX)$ with the category $\cB-\textup{Gr}$ of $\bZ^s$-graded $\cB$-modules. Let $\chi  = (\chi_1,\ldots,\chi_s) \in \bZ^s$ which can also be viewed as a character of $\bG_m^s$. By construction, the $\bG_m^s$-equivariant sheaf $\cO_P \otimes_k \chi$ is isomorphic to $\cL_1^{\otimes \chi_1} \otimes_{\bX} \otimes \cdots \otimes_{\bX} \cL_s^{\chi_s}$. Tensoring by $\cO_P \otimes_k \chi$ corresponds to the graded shift by $\chi$ operator $M \mapsto M[\chi]$  on $\cB-\textup{Gr}$. The push forward functor $c_* \colon \cB-\textup{Gr} \to \Qcoh(X)$ from the $\cB$-module viewpoint corresponds to taking the degree 0 part. Hence,
$$ c_* \left(\cL_1^{\otimes \chi_1} \otimes_{\bX} \otimes \cdots \otimes_{\bX} \cL_s^{\otimes \chi_s}\right) \simeq \cB_{\chi} $$
is a coherent sheaf on $X$. 

\begin{theorem}  \label{thm:tautological}
Let $\bX$ be a separated quasi-projective stack and $\cL_1,\ldots, \cL_s$ be line bundles on $\bX$ such that $\cL_1 \oplus \ldots \oplus \cL_s$ is faithful. Suppose that $\cG = \oplus \cG_i$ is a generator for $\bX$ with one summand, say $\cG_0$, of the form $c^* \cV \otimes_\bX \cL$ where $\cV$ is a vector bundle on $\cV$ and $\cL$ lies in the subgroup of the Picard group generated by $\cL_1,\ldots, \cL_s$. Then $\bM = \bM^{\cL_1,\ldots\cL_s}_{\cG-\textup{Sky}} \simeq \bX$.
\end{theorem}
\begin{proof}
To simplify notation, we write out the proof in the case that $s=2$. The general case is the same and can be obtained by inserting ellipses in appropriate places. 

To construct a morphism $\bX \to \bM$, it suffices to produce a flat family of tensor stable skyscraper sheaves over $\bX$. Proposition~\ref{prop:universalsky} shows that $\Delta_* \cO_{\bX}$ is a flat family of skyscraper sheaves over $\bX$. To show this family is stable under tensoring by $\cL_i$, note that Equation~\ref{eq:projsky} gives a natural isomorphism
$$ \theta_i \colon \cL_i \otimes_{\bX} \Delta_* \cO_{\bX} \xrightarrow{\sim} \Delta_*(\cL_i) \xrightarrow{\sim} \Delta_* \cO_{\bX} \otimes_{\bX} \cL_i .$$
The data of $\Delta_* \cO_{\bX}, \theta_1, \theta_2$ thus defines a morphism from $\bX \to \bM$. 

We now construct the inverse morphism $\Phi \colon \bM \to \bX$. Recall that $\bM$ is the stackification of a pre-stack $\bM^{pre}$ whose category of sections over a test scheme $T$ can be defined as follows. An object of $\bM^{pre}(T)$ consists of a flat family of skyscraper sheaves $\cS$ with respect to $\cL_1,\cL_2$ over $T$, and isomorphisms (expressed using Notation~\ref{notn:bimodtensor}) 
$\theta_i \colon \cL_i \otimes_{\bX} \cS \simeq \cS \otimes_T \cN_i$, for line bundles $\cN_1,\cN_2$ on $T$. Our assumption on $\cG$ and Proposition~\ref{prop:Vrankcoarse} ensures that $\cS$ has skyscraper $\cL$-rank, and the isomorphisms $\theta_i$ now ensure that $\cS$ also has skyscraper $\cL'$-rank for any $\cL' \in \langle \cL_1,\ldots, \cL_s \rangle$. 

We use Proposition~\ref{prop:XisPmodG} to view $\bX$ as the quotient stack $[P/\bG_m^2]$ where $P = \underline{\Spec}_X \cB$ for some $\bZ^2$-graded sheaf of algebras $\cB$ on $X$. 
If $\cB_T$ denotes the pullback of $\cB$ to $X \times T$, then we may view $\cS$ as a $\bZ^2$-graded $\cB_T$-module and the $\theta_i$ become isomorphisms of the form 
\begin{equation}
\theta_1 \colon \cS[1,0] \simeq \cS \otimes_T \cN_1, \quad 
\theta_1 \colon \cS[0,1] \simeq \cS \otimes_T \cN_2.
\label{eq:Sshiftstable}
\end{equation}
We wish to define the object $\Phi(\cS, \theta_1,\theta_2) \in \bX(T) = [P/\bG_m^2](T)$ which will be a diagram of the form
$$
\begin{CD}
\tilde{T} @>{f}>>  P \\
@V{\pi}VV @. \\
T @. 
\end{CD}
$$
where $\pi\colon \tilde{T} \to T$ is a $\bG_m^2$-torsor and $f\colon \tilde{T} \to P$ is a $\bG_m^2$-equivariant morphism. 
We define 
$$ \cO_{\tilde{T}} = \bigoplus_{\chi_1,\chi_2 \in \bZ} \cN_1^{\otimes \chi_1} \otimes_T \cN_2^{\otimes \chi_2} \quad \text{and} \quad  \tilde{T} = \underline{\Spec}_T \cO_{\tilde{T}}$$
which is naturally a $\bG_m^2$-torsor over $T$. 

To define $f$, we will first need to define the induced map on coarse moduli schemes $\bar{f} \colon T \to X$. The isomorphisms (\ref{eq:Sshiftstable}) ensure that the sheaves $\cS_{\chi_1\chi_2} \in \Coh(X \times T)$ are all isomorphic, so in particular, have the same support $Z \subseteq X \times T$. Now $\cS$ is a flat family of skyscraper sheaves relative to $\cG$ so we know the projection map $\phi \colon Z \to T$ is a finite map and, as remarked above, $\phi_* \cS_{i_1i_2}$ are line bundles on $T$. The canonical morphism of $\cO_T$-algebras $\cO_Z \xrightarrow{\rho} \underline{\cEnd}_{\cO_T} \cS_{00} \simeq \cO_T$ splits the identity on $\cO_T$. Now $\rho$ is injective by definition of support, so $\phi \colon Z \to T$ is an isomorphism. We may now define $\bar{f}$ to be the composite $\bar{f} = \pi_1 \phi^{-1}$ where $\pi_1 \colon Z \hookrightarrow X \times T \to X$ is projection onto the first factor. Note that $\cS$ is supported on the graph of $\bar{f}$ so, as a sheaf, is completely determined by $\bar{f}$ and its structure as a sheaf on $T$. 

We will define $f$ by constructing a morphism of $\bZ^2$-graded sheaves of algebras $\psi \colon \bar{f}^* \cB \to \cO_{\tilde{T}}$. Note first that the isomorphisms (\ref{eq:Sshiftstable}) show that 
$\cS \simeq \cS_{00} \otimes_T \cO_{\tilde{T}}$ as $\bZ^2$-graded sheaves on $T$. Let $\cE_{\chi}$ be the sheaf of (degree 0) graded homomorphism of sheaves $\cS \to \cS[-\chi]$ on $T$. Note that composition induces a natural algebra structure on $\cE =  \oplus_{\chi \in \bZ^2} \cE_{\chi}$. Furthermore, right multiplication on $\cS$ induces an injective  $\bZ^2$-graded homomorphism of sheaves of algebras $\cO_{\tilde{T}} \hookrightarrow \cE$. Left multiplication also induces a graded morphism of algebras $\psi \colon \cB \to \cE$, and will define our map $f$, once we show its image lies in $\cO_{\tilde{T}}$. This follows from the fact that the isomorphisms in (\ref{eq:Sshiftstable}) are isomorphisms of $\cB_T$-modules and the theory of endomorphisms compatible with shifts as explained in \cite[Section~3]{C12}. 

This completes the definition of $\Phi(\cS, \theta_1,\theta_2)$. It is now elementary, though tedious, to verify that a) this defines a morphism of pre-stacks $\bM^{pre} \to \bX$ and hence, morphism of stacks $\Phi \colon \bM \to \bX$, and that b) $\Phi$ is inverse to the morphism $\bX \to \bM$ we constructed using the universal skyscraper sheaf $\Delta_* \cO_{\bX}$. 
\end{proof}

\section{Tensor stable moduli of representations}\label{sc:tensor_stable}

In this section, we study equaliser stacks in the setting of representations of a finite dimensional algebra $A$. When $A$ is the endomorphism algebra of a tilting bundle $\cT$ on a separated smooth projective stack $\bX$, we exhibit $\bX$ as various equaliser stacks on the moduli of $A$-modules. The basic idea is to use tilting theory to construct a correspondence between moduli problems in $\Qcoh(\bX)$ with those in $\textup{mod}-A$.

We assume throughout that $A$ has finite global dimension. Fix a {\em dimension vector} $\vec{d} \in K_0(A)$. If we present $A$ as a quiver with relations so that $A \simeq kQ/I$ for some quiver $Q = (Q_0,Q_1)$ and some admissible ideal $I\triangleleft kQ$, then $\vec{d}$ can be viewed as an element of $\bZ^{Q_0}$. Let $\bM_{\vec{d}}$ denote the {\em rigidified moduli stack of $A$-modules with dimension vector $\vec{d}$}. It is an Artin stack of finite type an elementary description of which can be found in \cite[Section~2]{CL}.

Let $L_1,\ldots, L_s$ be two-sided tilting complexes on $A$, that is, they induce auto-equivalences 
$$- \otimes^L_A L_i \colon D^b_{fg}(A) \xrightarrow{\sim} D^b_{fg}(A)$$
where $D^b_{fg}(A)$ denotes the bounded derived category of finitely generated $A$-modules. This functor induces a partially defined map $\lambda_i \colon \bM_{\vec{d}} \dashto \bM_{\vec{d}}$ as follows. Let $\cM$ be a flat family of $A$-modules over a test scheme $T$ with dimension vector $\vec{d}$. By \cite[Proposition~3.3]{CL} (the proof given there for $L_i = DA$ works more generally in this setting), there is a locally closed subscheme $T' \subseteq T$ which is the locus where a) $H_p(\cM \otimes^L_A L_i) = 0$ for $p\neq 0$ and b), $H_0(\cM \otimes^L_A L_i)$ is flat over $T'$ with dimension vector $\vec{d}$. Furthermore, by \cite[Lemma~3.2]{CL}, we have
$$ H_p(\cM|_{T'} \otimes^L_A L_i) \simeq H_p(\cM \otimes^L_A L_i)|_{T'} .$$
As one varies $\cM$ and $T$, the locally closed subscheme $T'$ determines a locally closed substack $\bM_{\vec{d}}' \subseteq \bM_{\vec{d}}$. We conclude
\begin{proposition}  \label{prop:lambdamap}
The functor $- \otimes_A L_i \colon \textup{mod}-A \to \textup{mod}-A$ induces a partially defined map $\lambda_i \colon \bM_{\vec{d}} \dashto \bM_{\vec{d}}$ with domain of definition $\bM_{\vec{d}}'$ in the notation above. We let $\bM_{\vec{d}}^{L_1,\ldots,L_s}$ denote the fixed point stack of these partially defined maps and call it the {\em tensor stable moduli stack} of $A$-modules with dimension vector $\vec{d}$ with respect to $L_1,\ldots, L_s$. 
\end{proposition}

The algebras of interest are those arising from tilting theory. We thus suppose that $\bX$ is a separated smooth projective stack with a tilting bundle $\cT$ which we decompose into indecomposables $\cT = \oplus_{i \in Q_0} \cT_i$. The finite dimensional algebra $A = \End_{\bX}{\cT}$ has finite global dimension and has the form $A \simeq kQ/I$ where the vertex set of the quiver $Q$ is $Q_0$. We view $\cT$ as an $(A,\cO_{\bX})$-bimodule. 
\begin{proposition}  \label{prop:Tgenerates}
The tilting bundle $\cT$ is a generating sheaf for $\bX$.
\end{proposition}
\begin{proof}
Fix a geometric point $x \in \bX$. 
The stack $[\text{pt}/\Aut(x)]$ which may be presented using the Cartesian square 
\begin{equation*}
\begin{CD}
[\text{pt}/\Aut(x)] @>x>> \bX\\
@VVV @VVcV\\
\text{pt} @>x>> X.
\end{CD}
\end{equation*} 
The morphism $x$ is a closed embedding and hence $x \colon [\text{pt}/\Aut(x)] \rightarrow \bX$ is too. 
We then have that $x^* \circ x_* = \text{id}$ and so $x_*: D^b([\text{pt}/\Aut(x)]) \rightarrow D^b(\bX)$ is full and faithful. 
Let $\cV$ be a vector bundle on $[\text{pt}/\Aut(x)]$ given by some irreducible representation. 
Since $\Coh([\text{pt}/\Aut(x)])$ is semi-simple it suffices to show that $\Hom^\bullet(x^* \cT, \cV) \neq 0$. 
By adjuction we have $\Hom^\bullet(x^* \cT, \cV) = \Hom^\bullet(\cT, x_* \cV).$ 
Since $\cT$ generates the derived category we have $\Hom^\bullet(\cT, x_* \cV)= 0$ if and only if $x_* \cV = 0$. 
However this can not be true since that would imply $x^* \, x_* \cV = 0$ and hence $\cV =0$. 
\end{proof}
Let $\Phi = \RHom_{\bX}(\cT,-) \colon D^b_{c}(\bX) \to D^b_{fg}(A)$ denote the derived equivalence induced by $\cT$. 

Given line bundles $\cL_1,\ldots, \cL_s \in \Pic \bX$, we may thus consider the tensor stable moduli stack $\bM_{\cT}^{\cL_{\bullet}} := \bM_{\cT-Sky}^{\cL_1,\ldots,\cL_s}$. We now introduce the corresponding moduli stack on $\textup{mod}-A$ as follows. Firstly, we let $\vec{d}\in \bZ^{Q_0} = K_0(A)$ be defined by $d_i = \rank \cT_i$ so that for any skyscraper sheaf $\cS$ on $\bX$ relative to $\cT$, we have $\Phi(\cS)$ is an $A$-module with dimension vector $\vec{d}$. Consider the auto-equivalences
$$ \Phi \circ (\cL_i \otimes^L_{\bX} -) \circ \Phi^{-1} \colon D^b_{fg}(A) \to D^b_{fg}(A)$$
which by Rickard \cite{MR1002456} are naturally isomorphic to $- \otimes^L_A L_i$ for some two-sided tilting complexes $L_i$. We thus also have another tensor stable moduli stack $\bM_{\vec{d}}^{L_{\bullet}} := \bM_{\vec{d}}^{L_1,\ldots,L_s}$. 

We now examine how $\Phi = \RHom_{\bX}(\cT, -)$ induces a morphism $\phi \colon \bM_{\cT-Sky} \to \bM_{\vec{d}}$ and hence morphism $\bM_{\cT-Sky}^{\cL_{\bullet}} \to \bM_{\vec{d}}^{L_{\bullet}}$ which we also denote by $\phi$. Let $\cS$ be a flat family of skyscraper sheaves over $T = \Spec R$ relative to $\cT$ where $R$ is a noetherian ring. We use the bimodule Notation~\ref{notn:bimodtensor} for $\cS$ below. Following Grothendieck (see for example \cite[Chapter~III, Section~12]{Hart}), consider the functors 
$$ \phi^p := H^p(\cT^{\vee} \otimes_{\bX} \cS \otimes_R - ) \colon 
\textup{mod}-R \to \textup{mod}-(A \otimes R).$$
Now $\cT^{\vee}\otimes_{\bX} \cS$ has finite support over $R$ so $\phi^p = 0$ for $p >0$ and $\phi^0$ is exact. By \cite[Proposition~12.5]{Hart} and Remark~\ref{rem:Morita}, it follows that the natural transformation
$$ H^0(\cT^{\vee} \otimes_{\bX} \cS) \otimes_R (-) \to  \phi^0$$
is an isomorphism and hence $H^0(\cT^{\vee} \otimes_{\bX} \cS)$ is flat over $R$. Thus $\Phi$ is compatible with base change and we conclude 
\begin{proposition}  \label{prop:transfermap}
There is a well-defined morphism of stacks $\bM_{\cT-Sky} \to \bM_{\vec{d}}$ defined by the functor $\cS \mapsto \Phi(\cS) = H^0(\cT^{\vee} \otimes_{\bX} \cS)$ which induces the stack morphism $\phi \colon \bM_{\cT-Sky}^{\cL_{\bullet}} \to \bM_{\vec{d}}^{L_{\bullet}}$.
\end{proposition}

We wish now to show that the (quasi-)inverse functor 
$$\Psi:=(-) \otimes^L_{A} \cT \colon D^b_{fg}(A) \to D^b_{c}(\bX)$$
induces an inverse morphism $\psi \colon \bM_{\vec{d}}^{L_{\bullet}} \to \bM_{\cT-Sky}^{\cL_{\bullet}}$ to $\phi$. We need the following ampleness assumption: there is some tensor product $\cL^{\vec{i}} := \cL_1^{\otimes i_1} \otimes_{\bX} \ldots \otimes_{\bX} \cL_s^{\otimes i_s}$ such that the triple $(\Coh(\bX), \cT, (-)\otimes_{\bX} \cL^{\vec{i}})$ is ample in the sense of \cite{AZ94}. In this case we say more briefly that $(\cT, \cL_{\bullet})$ is {\em ample}.

\begin{theorem}  \label{thm:stackfromquiver}
Let $\cT = \oplus \cT_i$ be a tilting bundle on a smooth separated projective stack $\bX$ and $\cL_1, \ldots, \cL_s$ be line bundles such that $(\cT, \cL_{\bullet})$ is ample. We furthermore assume that one of the summands $\cT_i$ is isomorphic to $c^* \cV \otimes \cL$ for some vector bundle $\cV$ on $X$ and line bundle $\cL$ in the group generated by the $\cL_i$. Let $A = \End_{\bX} \cT$ and $L_1, \ldots, L_s$ be the two-sided tilting complexes above which correspond to $\cL_1,\ldots, \cL_s$. Then there is an isomorphism of stacks $\bX \simeq \bM^{L_{\bullet}}_{\vec{d}}$ where $\vec{d} \in K_0(A)$ is given by the rank vector of $\cT$. Furthermore, the isomorphism is given by the universal object defined by the bimodule $\ _{\cO_{\bX}}\cT^{\vee}_A$ together with some isomorphisms $ \cL_i \otimes_{\bX} \cT^{\vee} \simeq \cT^{\vee} \otimes_A^L L_i$. 
\end{theorem}
\begin{proof}
We first show that the morphism of stacks $\phi \colon \bM_{\cT-Sky}^{\cL_{\bullet}} \to \bM_{\vec{d}}^{L_{\bullet}}$ is an isomorphism in this case. Now $\phi$ is given by the equivalence $\RHom_{\bX}(\cT, -)$ so it suffices to show that the inverse equivalence $(-) \otimes^L_A \cT$ induces a well-defined morphism of stacks $\psi \colon \bM_{\vec{d}}^{L_{\bullet}} \to \bM_{\cT-Sky}^{\cL_{\bullet}}$. 

Let $T$ be a noetherian affine test scheme and consider an object of $\bM_{\vec{d}}^{L_{\bullet}}$ given by a flat family of $A$-modules $\cM$ over $T$ and isomorphisms $\theta_i \colon \cM \otimes_A^L L_i \xrightarrow{\sim} \cN_i \otimes_T \cM$ for some line bundles $\cN_i$ on $T$. Let $\cP := \cM \otimes_A^L \cT$ so for any $\vec{j} \in \bZ^s$ we have
$$\RHom_{\bX}(\cT, \cP \otimes \cL^{\otimes \vec{j}}) \simeq \cN^{\otimes \vec{j}} \otimes_T \cM \in \textup{Mod}-A.$$
Note that $\cP \in D^{\leq 0}$ is a bounded complex since $A$ has finite global dimension. Let $F_n = \Hom_{\bX}(\cT, - \otimes_{\bX} \cL^{\otimes n \vec{i}})$ where $\vec{i}$ is chosen so that $(\Coh(\bX), \cT, (-)\otimes_{\bX} \cL^{\vec{i}})$ is ample. Consider the hypercohomology spectral sequence is 
$$ R^pF_n H^q(\cP) \Rightarrow H^{p+q}(\RHom_{\bX}(\cT, \cP \otimes \cL^{\otimes n\vec{i}})).$$
For $n\gg 0$, the spectral sequence collapses to show that $F_n(H^q(\cP)) = 0$ for all $q \neq 0$. In particular, ampleness ensures that $\cP$ is concentrated in cohomological degree 0. Furthermore, the Serre module $\oplus_n F_n(\cP)$ is in sufficiently high degree equal to 
$\oplus \cN^{\otimes n \vec{i}} \otimes_T \cM$ which is flat over $T$. It follows since flatness is a property of the abelian category and \cite{AZ94}, that $\cP$ is flat over $T$ too. Furthermore, the Hilbert function of the Serre module is bounded so $\cP$ is even a flat family of finite length sheaves. The fact that it is also a family of skyscraper sheaves relative to $\cT$ follows from the fact that $\RHom_{\bX}(\cT, \cP) \simeq \cM$ which has the same rank vector as $\cT$. This completes the proof that $\psi$ is an isomorphism of stacks. Composing with the ``tautological'' isomorphism of Theorem~\ref{thm:tautological} gives the required isomorphism which maps the universal object $\Delta_{*}\cO_{\bX}$ (with appropriate isomorphisms) to $\cT^{\vee}$. 
\end{proof}

\section{Refined representation and their moduli}\label{sec:refined}

We introduce a slightly different approach for recovering our stack $\bX$ from a tilting bundle.
Here the focus will be on `fixing' the mismatch of monoidal structures between $\bX$ and $\bM$; after all \cite{Lurie} suggests that the tensor product holds the key to stack recovery.

The stack $\bM$ comes with a universal bundle $\cU  = \oplus_{i \in Q_0} \cU_i$.
Since the moduli stack $\bM$ is rigidified to remove the common $\bG_m$-stabiliser, $\cU$ is only uniquely defined up to twist by some line bundle on $\bM$.
The titling bundle $\cT^{\vee}$ gives a tautological family of quiver representations of $Q$ over $\bX$ and hence induces a morphism of stacks $f \colon \bX \rightarrow \bM$ such that $f^* \cU \simeq \cN \otimes_{\bX} \cT^{\vee}$ for some line bundle $\cN$ on $\bX$.
We will assume that one of the $\cT_0$ is $\cO_{\bX}$ for some $0 \in Q_0$ and further arrange matters so $\cU_0 = \cN = \cO_{\bM}$, hence $\cU$ and $f^*$ are well-defined.

The universal line bundles of $\bM$ generate a free abelian subgroup of $\Pic(\bM)$ of rank $|Q_0|-1$.
This may be checked by restricting to the `point' of $\bM$ corresponding to the semisimple module $\cM$ of dimension vector $\vecd$.
Given our assumption that $\cU_0 = \cO_{\bM}$, we may identify this group with a, the suggestively denoted, subgroup $\textup{Wt}(Q):= \Lambda_{Q_0 \setminus \{0\}} \subset \Lambda_{Q_0}$ generated by all the $\cU_i$ but $\cU_0$.
The homomorphism on $f^* \colon \Lambda_{Q_0} \rightarrow \Pic(\bX)$ is, in general, not an isomorphism even when restricted to $\textup{Wt}(Q)$; in general there exists pair of bundles $\cU_\alpha, \cU_\beta \in \textup{Wt}(Q)$ such that $\cU_\alpha \not\simeq \cU_\beta$ but for their respective pull-backs $\cT_\alpha \simeq \cT_\beta$.
The aim here is to tweak $\bM$ and the objects it parametrises so that the pullback homomorphism $f^*$ ends up being an isomorphism.
Naturally the kernel of the group homomorphism $f^*$ will play an important role in what follows we will use $\Lambda_r \subset \Lambda_{Q_0}$ to denote it.

\subsection{Definition of refined representations}
The first step in our construction is a natural $\Lambda_r^\vee$ cover $\tilde{\bM}$ of $\bM$ as was introduced in \cite{Abd}.
The universal bundle $\cU$ give us a natural morphism $\bM \rightarrow B\Lambda_{Q_0}^\vee$.
The subgroup $\Lambda_r \subset \Wt(Q) \subset \Lambda_{Q_0}$ in turn gives a morphisms $B\Lambda_{Q_0}^\vee \rightarrow B\Lambda_r^\vee$.
The composite gives $\bM \rightarrow B\Lambda_r^\vee$.
We define $\tilde{\bM}$ by the following 2-Cartesian square.
\begin{equation*}
\begin{CD}
\tilde{\bM} @>>> \bM \\
@VVV @VVhV \\
\text{pt} @>>> B\Lambda_r^{\vee} = [\text{pt}/\Lambda_r^{\vee}].
\end{CD}
\end{equation*}

We wish now to construct a near section of the $\Lambda_r^{\vee}$-cover $\tilde{\bM} \rightarrow \bM$.
First we pick a basis $B_r$ of $\Lambda_r$ and a choice of isomorphism $\gamma_{\kappa} \colon \cO \xto{\sim} \cT_{\kappa}^\vee$ for each $\kappa \in B_r$.
Let $\bfe_i \in A$ be the idempotent corresponding the the vertex $i$. 
Recall that the vector space $\bfe_i A \bfe_j$ is generated by paths from a vertex $i$ to another $j$ and is isomorphic to the space of sections $\Hom_\bX (\cT_i^\vee, \cT_j^\vee)$. 
Therefore given two pairs of line bundles $(\cT_i, \cT_j)$ and $(\cT_k, \cT_l)$ for which $\cHom_\bX(\cT^\vee_i, \cT_j^\vee) \cong \cHom_\bX(\cT_k^\vee, \cT_l^\vee)$ our choice of $\gamma$ induces a natural isomorphism $\gamma \colon \bfe_i A \bfe_j \xrightarrow{\sim} \bfe_k A \bfe_l$.
Below, we view $A$-modules $\cM$ as quiver representations $(\cM_i,m_a)$ where the indices range over $i \in Q_0, a \in Q_1$. 

\begin{definition} \label{def:refined}
A {\em flat family of refined representations} $\cM$ of $A$ over $S$ of dimension vector $\vec{d}$ consists of a representation $(\cM_i, m_a, g) \in \tilde{\bM}(S)$ such that $g$ satisfies the following condition: for any two pairs of vertices $(i,j)$ and $(k,l)$ for which $\chi:= (\chi_i + \chi_l)-(\chi_j + \chi_k) \in \Lambda_r$ and path $a \colon i \rightarrow j$, the following diagram commutes
\begin{equation}  \label{eq:refined}
\begin{CD}
\cM_i \otimes \cM_k @>m_a \otimes \text{id}>> \cM_j \otimes \cM_k\\
@VV\text{id}V @VVg(\chi)V\\
\cM_i \otimes \cM_k @>\text{id} \otimes m_{\gamma(a)}>> \cM_i \otimes \cM_l.
\end{CD}
\end{equation}
We let $\bM_{\text{ref}}$ denote the resulting {\em moduli stack of refined representations of $A$}. We refer to $g$ as {\em refinement data}.
\end{definition}

Our choice of isomorphisms $\gamma_\kappa$ for every $\kappa \in B_r$ gives a family of refined representations over $\bX$ and so a morphism $\bX \rightarrow \bM_\text{ref}$.
The change of basis matrix between any two such choices give a 2-isomorphism between the associated morphisms $\bX \rightarrow \bM_\text{ref}$.

\begin{remark}
This definition of refined representations is slightly different from that given in \cite[Definition 3.2]{Abd} where the commutative diagram condition is omitted.
\end{remark}

\subsection{Monoidal interpretation of refinement data}
There are several monoidal categories of interest here. 
Firstly, for any stack $S$, we let $\Vect_1 (S)$ denote the symmetric monoidal category of line bundles on $S$, where the morphisms are the isomorphisms. 
The category is also rigid in the sense that it has (left and right) duals. 

Secondly, given an abelian group $\Lambda$ and subgroup $\Lambda'$, we define a symmetric monoidal category $\underline{\Lambda}/\Lambda'$ whose objects are the elements of $\Lambda$.
The morphisms are given by a pair $\lambda \in \Lambda, \lambda' \in \Lambda'$ and have the form $\lambda \xto{+\lambda'} \lambda + \lambda'$. 
Composition of morphisms is given by addition. The tensor product is also given by addition whilst the braiding and the associator are given by the identity $+0$. Note that morphisms in this category are unique (if they exist). 
This observation is useful to keep in mind when verifying diagrams in $\underline{\Lambda}/\Lambda'$ commute. 
In particular, it streamlines checking that $\underline{\Lambda}/\Lambda'$ is a symmetric monoidal category, an elementary verification we omit. Note also that it is rigid with duals given by negatives.
When $\Lambda' = 0$ we write $\underline{\Lambda} = \underline{\Lambda}/\Lambda'$.
The category $\underline{\Lambda}/\Lambda'$ is monoidally equivalent to $\underline{\Lambda/\Lambda'}$.

A family of representations $(\cM_i,  m_a)$ of $A$ over a test scheme $S$ gives a monoidal functor $\underline{\Lambda}_{Q_0} \rightarrow \Vect_1(S)$ taking the generators to their corresponding bundles $\cM_i$.
The refinement data $g$ is equivalent to giving a lift of this functor to $\underline{\Lambda}_{Q_0}/\Lambda_r$ resulting in a functor $\underline{\Lambda}_{Q_0}/\Lambda_r \rightarrow \Vect_1(S)$.
Our choice of $\gamma(\kappa)$ for every $\kappa \in B_r$ (and by extension $\kappa \in \Lambda_r$) is precisely a lift of the functor $\underline{\Lambda}_{Q_0} \rightarrow \Vect_1(\bX)$ taking $\chi_i \mapsto \cT_i^\vee$ to a functor $\underline{\Lambda}_{Q_0}/\Lambda_r \rightarrow \Vect_1(\bX)$.

\subsection{From Picard stable to refined representations}
Now we address how a point in the Picard stable moduli space $\bM^{\Pic}$ naturally induces refined data on the corresponding representation $\bM$.
We seek a morphism $h \colon \bM^{\Pic} \to \bM_{\text{ref}}$ assigning refinement data to tensor stability data.

Consider a flat family of $A$-modules $\cM \in \bM^{\Pic}(S)$ over a test scheme $S$.
The tensor stability data consists of compatible isomorphisms 
$$\psi_i \colon \cM \otimes_A^L L_i \xto{\sim} \cN_i \otimes_S \cM$$
where $\cN_i$ are line bundles on $S$. 
The isomorphisms $\psi_i$ are only defined up to scalar and we will need to show at the end, that our definition of $h(\cM,\psi_i)$ is independent of this ambiguity.

There are a several functors from $\underline{\Lambda}_{Q_0}$ to $\Vect(S)$ that are in play here: there are $\cM_?$ and $\cN_?$, which were defined above, and then $L_?$ and $\cM_?^{\text{can}}$ which we will define below.
The functors $\cM_?$ and $\cN_?$ are monoidal while $L_?$ and $\cM_?^{\text{can}}$ turn out not to be monoidal.
The refinement data corresponding to $\psi_i$ will be derived by relating these functors to each other.

\begin{lemma}\label{lm:equivNM}
The stability data gives a natural isomorphism of monoidal functors  $\xi \colon \cM_? \rightarrow \cN_?$.
\end{lemma}

\begin{proof}
First note that for $i\in Q_0$ we have that $(\cM \otimes_A^L L_i)_0 = \cM_i$.
Furthermore $\cM_0 = \cO_S$ for our special vertex $0\in Q_0$.
Our stability data then gives
$$\cM_i = (\cM \otimes_A^L L_i)_0 \xto{\psi_{i,0}} \cN_i \otimes_S \cM_0 = \cN_i.$$
Tensoring over $S$ induces the required isomorphisms $\xi_\chi \colon \cM_\chi \xto{\sim} \cN_\chi$.
\end{proof}

The other two functors of interest $L_?$ and $\cM_?^{\text{can}}$ are given as follows.
The functor $L_?$ takes the value  $(\cM \otimes_A^L L_\chi)_0$ for $0 \in Q_0$ our special vertex and $\chi \in \Lambda_{Q_0}$ a general element.
For $\cM_?^{\text{can}}$, let $K^b(P_i)$ be the homotopy category of complexes whose components are direct sums of the projectives $P_i := \bfe_iA$, we begin by picking a quasi-inverse of the triangulated equivalence given by the natural inclusion $K^b(P_i) \rightarrow D^b(A)$.
Given $\chi \in \Lambda_{Q_0}$, the image of the module $L_\chi$ under this functor $D^b(A) \rightarrow K^b(P_i)$ is by definition a complex whose components are direct sums of projectives of the form $\bfe_iA$.
We then have an expression of $(\cM \otimes_A^L L_\chi)_0$ as a complex whose components are direct sums of $\cM_i$.
Take $\cM_?^{\text{can}}$ to be the determinant of this complex of $S$-bundles.

\begin{lemma}\label{lm:Mcan}
The functors $(\cM \otimes_A^L L_\chi)_0$ and $\cM_?^{\textup{can}}$ are canonically isomorphic.
\end{lemma}

\begin{proof}
The complex $(\cM \otimes_A^L L_\chi)_0$ is a complex of bundles whose cohomology is a line bundle concentrated in one degree. 
There is a canonical isomorphism from this to its determinant $\cM_?^{\text{can}}$.
\end{proof}

For $i \in Q_0$, $\cM_i= \cM_i^\text{can}$ but this is not true for general $\chi \in \Lambda_{Q_0}$ i.e.\ in general, $\cM_\chi \neq \cM_\chi^\text{can}$.
Stability data is designed to `fix' this discrepancy.

\begin{example}
Take the weighted projective line $\bX:=\bP(1,2)$ and $\cT= \oplus_{i=0}^2 \,\cO(i)$.
Let $(\cU_i, u_a)$ be the universal family on moduli space of quiver representations $\bM$.
For $i=1$, we have $(\cU \otimes_A^L \cO(1))_0 = \cU_1$.
For $\chi= 2\,\chi_1$, $$\cU_\chi^\text{can} = (\cU \otimes_A^L (\cO(1)\otimes_\bX \cO(1)))_0 = (\cU \otimes_A^L (\cO(2))_0 = \cU_2.$$
On the other hand, $\cU_\chi = \cU_1 \otimes_\bM \cU_1$ and $\cU_2 \not\simeq \cU_1 \otimes_\bM \cU_1$ on $\bM$.
\end{example}

\begin{proposition}\label{prop:can}
Stability data gives an isomorphism of functors between $\cM_?$ and $\cM_?^{\textup{can}}$ that is independent of scaling. 
\end{proposition}

\begin{proof}
This is essentially a translation of Proposition~\ref{prop:doublestable}.

For $\chi \in \Lambda_{Q_0}$ Lemmas~\ref{lm:equivNM} and \ref{lm:Mcan} give us
\begin{equation}\label{eq:phitog}
\cM_\chi^\text{can} \xto{\sim} (\cM \otimes_A^L L_\chi)_0 \xto{\psi_{\chi,0}} \cN_\chi \otimes_S \cM_0 = \cN_i \xto{{\xi_\chi^{-1}}} \cM_\chi.
\end{equation}
Scaling the stability data $(\psi_i)_{i\in Q_0}$ multiplies $\psi_\chi$ and $\xi_\chi$ by the same scalar for all $\chi \in \Lambda_{Q_0}$ and hence the result.
\end{proof}

\begin{lemma}\label{lm:kappa0}
Take $\kappa \in \Lambda_r$ then $\cM_0 = \cM_\kappa^\textup{can}$.
\end{lemma}

\begin{proof}
The class of $L_\kappa$ in the Grothendieck group of $A$ is equal to that of $L_0$ since they are isomorphic.
Therefore the determinant of the projective resolution of $L_\kappa$ must be equal to $L_0$. Otherwise it would give a non-trivial relation in the Grothendieck group of $A$ which is freely generated by the classes of $L_i$ for $i\in Q_0$.
\end{proof}

Putting Proposition~\ref{prop:can} and Lemma~\ref{lm:kappa0} together we have isomorphisms $$g_\kappa \colon \cM_0 = \cM_\kappa^{\text{can}} \rightarrow \cM_\kappa$$
for every $\kappa \in \Lambda_r$.
These give well-defined refined data on $\cM$  by Proposition~\ref{prop:doublestable}.
We therefore have a morphism $\bM^{\Pic} \rightarrow \bM_\text{ref}$.

\subsection{From refined representations to Picard stable}

We first identify a locally closed subset of $\bM_\text{ref}$ that will feature it what follows.
There is a forgetful morphism $\bM_\text{ref} \rightarrow \bM$ that ignores refinement data.
We may then, as before, consider the locally closed substack $\bM'$ of $\bM$ where all the $-\otimes_A L_l \colon \bM \dashto \bM$ are defined.
We will use $\bM_\text{ref}'$ to denote the locally closed subset of $\bM_\text{ref}$ that maps to $\bM'$ under the forgetful map.
Note that the image of $\bM^{\Pic} \rightarrow \bM_\text{ref}$ lies in $\bM_\text{ref}'$.

We will work with universal family on $\bM_\text{ref}'$: this is the universal line bundles $\cU_i$ for $i \in Q_0$, universal sections $u_a$ for $a \in Q_1$ and a lift $g \colon \underline{\Lambda}_{Q_0}/\Lambda_r \rightarrow \Vect(\bM)$ of the functor $\cU_? \colon \underline{\Lambda}_{Q_0} \rightarrow \Vect(\bM)$.
From this we aim to define stability data $\psi_i \colon \cU \otimes_A^L L_i \rightarrow \cN_i \otimes_\bM \cU$ for $i\in Q_0$ and then a morphism $\bM_\text{ref}' \rightarrow \bM^{\Pic}$.
We do this componentwise defining isomorphisms of sheaves $\psi_{i,j} \colon (\cU \otimes_A^L L_i)_j \rightarrow \cU_i \otimes_\bM \cU_j$.

Our triangulated equivalence to $K^b(P_i)$ gives us an expression of $(\cU \otimes_A^L L_i)_j$ as a complex whose components are direct sums of the bundles $\cU_i$.
Moreover, since our family is in $\bM_\text{ref}'$ the cohomology of this complex is a line bundle concentrated in degree zero for every $i,j \in Q_0$.
Therefore the functor $\cU_?^\text{can}$ is well defined and thus $(\cU \otimes_A^L L_i)_j$ is canonically isomorphic to $\cU_{\chi}$ for some $\chi \in \Lambda_{Q_0}$.
The line bundle $\cU_i \otimes_\bM \cU_j$ is also of this form, it is $\cU_{\chi_i+\chi_j}$.

We compare our bundles $(\cU \otimes_A^L L_i)$ and $\cU_i \otimes_S \cU$ under pullback by the morphism $f \colon \bX \rightarrow \bM$: we have natural isomorphisms
\begin{equation} \label{eq:natisolinebdls}
f^*(\cU \otimes_A^L L_i) \xto{\sim} f^* \cU \otimes_A^L L_i \xto{\sim} 
\cT^{\vee} \otimes_A^L L_i \xto{\sim} \cT_i \otimes_{\bX} \cT^{\vee} \xto{\sim} 
f^*\cU_{i}^{\vee} \otimes_{\bX} f^* \cU.
\end{equation}
This implies that the images of $\cU_\chi:= \cU_{\chi_i + \chi_j}^\text{can} \simeq (\cU \otimes_A^L L_i)_j$ and $\cU_i \otimes_\bM \cU_j$ under the composite $$\underline{\Lambda}_{Q_0} \xto{\cU_?} \Vect(\bM) \xto{f^*} \Vect(\bX)$$ are isomorphic.
Hence, by definition of $\Lambda_r$, the element $\chi-(\chi_i+\chi_j) \in \Lambda_r$.
The isomorphism $\psi_{i,j}$ is then defined by the composite:
\begin{equation}\label{eq:gtophi}
(\cU \otimes_A^L L_i)_j \xto{\sim} \cU_\chi = \cU_{\chi_i + \chi_j}^\text{can} \xto{g_{\chi-(\chi_i+\chi_j)}} \cU_{\chi_i + \chi_j} = \cU_i \otimes_\bM \cU_j.
\end{equation}
The isomorphisms $\psi_{i,j}$ for varying $j \in Q_0$ assemble to give the desired $A$-module isomorphism $\psi_i$. 
Indeed the commutative diagrams in Definition~\ref{def:refined} ensure compatibility with the $A$-module structure. 
This completes the construction of the morphism $\bM'_{\text{ref}} \rightarrow \bM^{\Pic}$.


\begin{theorem}
The morphisms are $\bM'_{\textup{ref}} \rightarrow \bM^{\Pic}$ and $\bM^{\Pic} \rightarrow \bM_\textup{ref}'$ are mutual quasi-inverses and give a isomorphism of stacks. 
\end{theorem}

\begin{proof}
Observe that $(\cU \otimes_A^L L_{i+j})_0 = (\cU \otimes_A^L L_i)_j$.
Start with refinement data $g_\kappa$ for $\kappa in \Lambda_r$.
Then substituting Equation~\ref{eq:gtophi} in Equation~\ref{eq:phitog} gives that $\bM'_{\textup{ref}} \rightarrow \bM^{\Pic} \rightarrow \bM_\textup{ref}'$ is the identity.
Similarly starting with tensor stability data $\phi_\chi$ for $\chi \in \Wt(Q)$, substituting Equation~\ref{eq:phitog} in Equation~\ref{eq:gtophi} gives that $\bM'_{\textup{ref}} \rightarrow \bM^{\Pic} \rightarrow \bM_\textup{ref}'$ is the identity.
\end{proof}

\begin{corollary}\label{cor:mref}
The stack $\bM_{ref}'$ is isomorphic to $\bX$.
\end{corollary}

\section{Global quotient presentation of $\bM_\textup{ref}$}\label{sc:GIT}

The moduli space of refined representations, as in the `not-refined' version, has a natural global quotient presentation.
Given a refined quiver representation $(\cM_i, m_a, g_\kappa)$ over $k$, let $\text{Func}_{\otimes, \cM_i}(\underline{\Lambda}_{Q_0}/\Lambda_r, \Vect(k))$ denote the set of categorifications of $\Lambda_r$ relations of the functor $\underline{\Lambda}_{Q_0} \rightarrow \Vect(k)$ taking $i \in Q_0$ to $\cM_i$.
Picking a basis for $\cM_i$ to identify them with $k$ enables us to view the $m_a$ as elements of $\bA^1$ and $g$ as elements of $\Lambda_r^{\vee}$
We thus obtain an element of
\begin{equation*}
\cR(Q, \vec{d}) := \oplus_{a \in Q_1} \Hom(\cM_{t(a)}, \cM_{h(a)}) \times \text{Func}_{\otimes, \cM_i}(\underline{\Lambda}_{Q_0}/\Lambda_r, \Vect(k))\simeq \bA^{Q_1} \times \Lambda_r^{\vee}. \end{equation*}
We use $\cR_A$ to denote the {\em refined representation space} which is the closed subscheme of $\cR(Q, \vec{d})$ whose elements descend to representations of $A \simeq kQ/I$ and furthermore make the diagrams in (\ref{eq:refined}) commute.
The gauge group
\begin{equation*}
\GL(\vec{d}) := \oplus_{i \in Q_0} \GL(\cM_i) \simeq \bG_m^{Q_0}.
\end{equation*}
naturally acts on $\cR_A$ by change of basis.
Furthermore the group $\Lambda_{Q_0}$ can be naturally identified with its group of characters.
Note that the diagonal one-parameter subgroup 
$$\Delta = \{(\lambda, \ldots,\lambda) | \lambda \in \bG_m\} \leq \GL(\vec{d})$$ 
acts trivially so there is an induced action of $\PGL(\vec{d}) := \GL(\vec{d}) / \Delta$.
Taking the quotient by $\PGL(\vec{d})$ as opposed to $\GL(\vec{d})$ amounts to considering the rigidified moduli space as opposed to the unrigidified version. 
An argument similar to the proof of \cite[Proposition 3.9]{Abd} shows that $\bM_{\text{ref}}$ is isomorphic to the quotient stack $[\cR_A/ \PGL(\vec{d})]$.

\subsection{$\bM_{\text{ref}}'$ and GIT stability}

We'll attempt to study the locus $\bM_{\text{ref}}'$, i.e.\ the one isomorphic to $\bX$, using GIT.
First we make the following observation.

\begin{lemma}\label{lm:open}
The subset $\bM_{\textup{ref}}' \subset \bM_{\textup{ref}}$ is open.
\end{lemma}

\begin{proof}
Fix a pair of vertices $i,j$ and consider the object $(\cU \otimes^L_A L_j)_i \in D^b(\bM_\text{ref})$ where $\cU$ is the universal family on  $\bM_{\text{ref}}$. This is a complex is of rank one.
First note that when restricted to the image of $\bX$, this complex is concentrated in degree zero and is quasi-isomorphic to the line bundle $\cT_i^{\vee} \otimes_{\bX} \cL_j$. This implies that the cohomology sheaves at nonzero degrees are torsion and furthermore, that the cohomology sheaf in degree zero is of rank one.
The nonzero cohomology sheaves and the torsion part of the zero cohomology sheaf are supported on a closed set that does not contain the image of $X$. The complement of this subset is $\bM_{\text{ref}}'$ and hence the result.
\end{proof}

Lemma~\ref{lm:open} hints that perhaps $\bM'_{\textup{ref}}$ is carved out by some GIT stability parameter. 

As discussed above, $\bM_{\text{ref}}$ is isomorphic to the quotient stack $[\cR_A/ \PGL(\vec{d})]$.
Following King \cite{Ki}, one may define an intrisic notion of $\theta$-stability for refined representations equivalent to the GIT stability of the $\PGL(\vec{d})$ action on $\cR_A$,  see \cite[Definition 3.4]{Abd}. 
Observe that $\PGL(\vec{d})$ maybe identified with the subgroup $(1 \times \oplus_{i\neq 0} \GL(\cM_i)) \subset \GL(\vec{d})$.
This in turn identifies $\textup{Wt}(Q)$ with the characters of $\PGL(\vec{d})$ so we may consider the $\theta$-semistable points in $\cR_A$ for $\theta \in \textup{Wt}(Q)$.
We will use $\bM_{\textup{ref}}^\theta$ to denote the semistable locus of $\bM_{\textup{ref}}$.


\begin{definition}
Given a generic stability parameter $\theta \in \textup{Wt}(Q)$ we say {\em $\theta$ stabilises $\bX$} if $f(x)$ is $\theta$-stable for all $x \in \bX$.
In other words, {\em $\theta$ stabilises $\bX$} if $f \colon \bX \rightarrow \bM_{\textup{ref}}$ factors through $\bM_{\textup{ref}}^\theta$.
\end{definition}

\begin{theorem}\label{thm:stab}
Let $\theta \in \textup{Wt}(Q)$ be a stability parameter that stabilises $\bX$ then the stacks $\bM_{\textup{ref}}' \simeq \bX$ are isomorphic to a connected component of $\bM_{\textup{ref}}^\theta$.
\end{theorem}

\begin{proof}
The fact that $\theta$ stabilises $\bX$ gives a morphism $f \colon \bX \rightarrow \bM_{\text{ref}}^\theta$.
This combined with Corollary~\ref{cor:mref} and Lemma~\ref{lm:open} gives that $f$ is an open embedding of $\bX$ in $\bM_{\text{ref}}^\theta$.

We also have that the image of the coarse moduli space $X$ under the coarse moduli map induced by $f$ is closed in the coarse moduli space of $\bM_\text{ref}^\theta$. 
Since $\bX$ is embedded in $\bM_\text{ref}^\theta$ this implies that $f$ is a also a closed embedding.
The result follows.
\end{proof}

\begin{remark}
The existence of a $\theta$ that stabilises $\bX$ is part of the hypothesis of Theorem~\ref{thm:stab}.
We expect such $\theta \in \textup{Wt}(Q)$ to exist in general.
In fact, in Lemma~\ref{lm:generic} below, we earmark a candidate.
\end{remark}

\begin{lemma}\label{lm:generic}
There exists a stability condition $\theta \in \textup{Wt}(Q)$ so that $\theta$ is generic, i.e.\ $\theta$-semistable implies $\theta$-stable, and $\textup{pic}(\theta)$ is the pullback of a very ample line bundle on the coarse moduli space $X$ of $\bX$.
\end{lemma}

\begin{proof}
Definition 3.4 of \cite{Abd} only tests $\theta \in \Lambda_{Q_0}$ against filtrations $\cM_\bullet$ of $(\cM_i, m_a)$ that satisfy $\kappa(W_\bullet)=0$ for all $\kappa\in \Lambda_r \subset \Lambda_{Q_0}$.
In other words, stability is only dependent on the class of $\theta$ in $\Lambda_{Q_0}/\Lambda_r$.
Now $\cT$ generates $D^b(\bX)$ so the homomorphism $\text{pic}\colon \Lambda_{Q_0} \rightarrow \Pic(\bX)$ is surjective and we have an isomorphism $\Lambda_{Q_0}/\Lambda_r \simeq \Pic(\bX)$.
Furthermore, ampleness is a generic condition in $\Pic(X)_\bQ \simeq \Pic(\bX)_\bQ$.
Therefore, if necessary, one may perturb $\theta$ so that it is generic and $\text{pic}(\theta)$ is pulled back from the ample cone of $X$.
\end{proof}

\begin{remark}\label{rm:generic}
The statement corresponding to Lemma~\ref{lm:generic} is not true when applied to the moduli of quiver representations of the tilting quiver of a general projective DM stack.
\end{remark}

\subsection{An especially nice situation}\label{ssec:nice}
In this subsection we investigate a case when the isomorphism $\bX \simeq \bM_\text{ref}^\theta$ is more direct than has been discussed so far.
From here on we will assume that $\bX$ is a Mori-dream stack, i.e.\ that $\bX$ has a finitely generated Cox ring $R$ and that $$\bX \simeq \bigg[\frac{\Spec(R) \setminus V(B_\theta)}{\Pic(\bX)^\vee}\bigg]$$ for $B_\theta$ the irrelevant ideal given by some ample line bundle $\theta \in \Pic(\bX)$.
We take $S$ to be the Cox ring of the moduli space $\bM$. The ring $S$ is naturally graded by $\text{Wt}(Q)$; the graded component $S_{j-i}$ has a natural $k$-basis given by $\bfe_i A \bfe_j$.
Note that $\bM \simeq [\Spec(S)/\PGL(\vec{1})]$. 
The morphism $f \colon \bX \rightarrow \bM$ is induced from tautological homomorphism of Cox rings $g \colon S \rightarrow R$ with the group homomorphism $f^* \colon \Wt(Q) \rightarrow \Pic(\bX)$ intertwining the grading.

As discussed just above Definition ~\ref{def:refined}, for any two pairs of elements $(i,j)$ and $(k,l)$ of $Q_0$ for which $(\chi_i + \chi_l)-(\chi_j + \chi_k) \in \text{ker}(f^*)$ we have an isomorphism of Peirce components $\gamma(i,j;k,l) \colon \bfe_i A \bfe_j \xrightarrow{\sim} \bfe_k A \bfe_l$. 
This induces an isomorphism $\gamma(i,j;k,l) \colon S_{j-i} \xrightarrow{\sim} S_{l-k}$.
Hence graded components of the Cox ring $R$ may become `separated' in the $Q_0$ indexed algebra $A$.
This separation of components gives rise to some obvious elements of $\ker g$ that have the form 
\begin{equation}  \label{eq:deindex}
    y - \gamma(i,j;k,l) (y)
\end{equation}
where $y \in S_{j-i}$ and $i,j,k,l \in Q_0$ are such that $(\chi_i + \chi_l)-(\chi_j + \chi_k) \in \text{ker}(f^*)$. 
We let $I_{\textup{de}}\subseteq \ker g$ be the $\Wt(Q)$-graded ideal generated by these elements.

\begin{definition}
We say our collection {\em captures the Cox ring of $\bX$} if $g$ is surjective descends to an isomorphism $S/I_{de} \simeq R$.
\end{definition}

In the following we demonstrate how the isomorphism $\bX \simeq \bM_{\text{ref}}^\theta$ can be obtained from the ring homomorphism $g$ when $\cT$ captures the Cox ring of $\bX$.
To further clarify, this is precisely when the difference between $S$ and $R$ is only due to the separation of Peirce components discussed above.

The short exact sequence of abelian groups
\begin{equation} \label{eq:defineF}
     0 \to \Lambda_r \to \Wt(Q) \to \Pic(\bX) \to 0
\end{equation}
will, as always, play a crucial role.
The group $\PGL(\vec{1})$, which is $\Wt(Q)^\vee$, acts diagonally on $\Spec S \times \Lambda_r^\vee$ so $$[\Spec R/\Pic(\bX)^\vee] \simeq [V/\PGL(\vec{1})]$$ where $V$ is the $\PGL(\vec{1})$-orbit of $\Spec R \times 1 \subseteq \Spec S \times \Lambda_r^\vee$.
The refined representation space $\cR_A$ is a closed subscheme of $\Spec S \times \Lambda_r^\vee$ cut-out by the commutative diagrams (\ref{eq:refined}).
We'll eventually show that in this setting $\cR_A = V$.

We clarify what $V$ is. 
Let $\frakm\triangleleft k\Lambda_r$ be the maximal ideal corresponding to $1 \in \Lambda_r^\vee$. 
Then $V$ is the closed subscheme defined by the ideal sheaf
$$ I_V := \bigcap_{\tau \in \PGL(\vec{1})} \tau\cdot\left(I \otimes k\Lambda_r + S \otimes \frakm \right)$$
To compute this, we need some notation. 
Recall that $I$ is only $\Pic(\bX)$-homogeneous. Let $s \in I$ be a $\Pic(\bX)$-homogeneous element. 
Then we can express it as a sum of $\Wt(Q)$-homogeneous elements $s = s_1 + \ldots + s_m$ all of whose degrees lie in the same coset of $\Lambda_r$. 
Hence we can pick $\kappa_i \in \Lambda_r$ such that $s^h := s_1 \otimes \kappa_1 + \ldots + s_m \otimes \kappa_m$ is $\Wt(Q)$-homogeneous. 
The {\em homogenisation} $s^h$ is only defined up to multiplication by some $\kappa \in \Lambda_r$. 

\begin{lemma}
Let $\Sigma \subset I$ be a set of $\Wt(Q)$-homogeneous generators of $I$ then $I_V = \langle s^h \,|\, s \in \Sigma \rangle.$ 
\end{lemma}

Under the assumption that $\cT$ captures the Cox ring of $\bX$ we have that $I$ is generated by elements of the form $s:= y - \gamma (y)$ for some $\Wt(Q)$-homogeneous element $y \in S$.
Its homogenisation $s^h$ can be encoded moduli-theoretically on the refined representation as follows.
Take $(\cM_i, m_a, g)$ a $k$-point of $\bM_\text{ref}$.
Suppose that $i,j,k,l \in Q_0$ are such that $\kappa:= (\chi_i + \chi_l)-(\chi_j + \chi_k) \in \Lambda_r$. 
The refinement data $g$ thus gives an isomorphism $g_{\kappa} \colon \Hom_k(\cM_i,\cM_j) \to \Hom_k(\cM_k,\cM_l)$. 
Suppose $a \in \bfe_i A \bfe_j$ so we have a  multiplication by $a$ map $m_a \colon \cM_i \to \cM_j$ and multiplication by $\gamma(a)$ map $m_{\gamma(a)} \colon  \cM_{k} \to \cM_{l}$. Then the relation $s^h$ corresponds to commutativity of the following diagram. 
\begin{equation}
\begin{CD}
k @>m_a \otimes \cM_{i}^*>> \cM_{j} \otimes \cM_{i}^*\\
@VV\text{id}V @VV{g_\kappa}V\\
k @>m_{\gamma(a)}\otimes \cM_{k}^*>> \cM_{l} \otimes \cM_{k}^*
\end{CD}
\end{equation}
This is precisely the tensor product of the diagram (\ref{eq:refined}) by $(\cM_i \otimes \cM_k)^*$.
Hence $\cR_A = V$.

\begin{corollary}
The ring homomorphisms $g$ along with the intertwining homomorphism $f^*$ give an isomorphism of stacks $$\bigg[\frac{\Spec R}{\Pic(\bX)^\vee}\bigg] \simeq \bigg[\frac{\cR_A}{\PGL(\vec{1})}\bigg].$$
\end{corollary}

\begin{corollary}
There is a GIT parameter $\theta \in \Wt(Q)$ for which $\bX \simeq \bM_\textup{ref}^\theta$.
\end{corollary}


\section{An example: G-L projective spaces}\label{sec:HIMO}

Here we consider Geigle-Lenzing (G-L) projective space as defined by Herschend, Iyama, Minamoto, Oppermann in \cite{HIMO}.
First we begin by briefly recalling the construction in \cite{HIMO}. 

The building blocks for G-L projective spaces are: a polynomial ring $C:= k[t_0, \ldots, t_d]$ and $n$ linear forms $l_0,\ldots,l_n \in C$ each with an assigned weight $p_i \in \bN$. 
The forms are required to be in general position, i.e.\ every subset of at most $d+1$ forms is linearly independent. 
Define $$R':= k[t_0,\ldots, t_d, y_0, \ldots, y_n],  \quad I := \langle y_i^{p_i} - l_i \, | \, 0\leq i \leq n \rangle \subset R'$$ and take $R:= R'/I$. 
The ring $R$ is graded by the abelian group $$\bL:= \bZ \vecy_0 \oplus \cdots \oplus \bZ \vecy_n \oplus \bZ \vecc \,/\, \langle p_i \vecy_i - \vecc \, | \, 0\leq i \leq n \rangle$$ with $\deg(t_i)=\vecc$ and $\deg(y_i) = \vecy_i$. 
The corresponding stack is then given by $$\bX:= \bigg[\frac{\Spec(R) \setminus \{0\}}{\bL^\vee}\bigg].$$ 
The Picard group of $\bX$ is given by $\bL$ and has a partial ordering defined by $$\vecy \leq \vecz \iff \Hom(\vecy, \vecz) \neq 0.$$

The ring $R$ may be expressed in a slightly different fashion. 
We may assume that $n\geq d$ by adjoining an indeterminant $y_i$ for every extra $t_i$ variable and setting $l_i= t_i$ and $p_i=1$. 
Now the genericity assumption allows us to change variables so that the first $d+1$ forms $l_i$ are just $t_i$.
The ring $R$ is then naturally isomorphic to a quotient of $k[y_0, \ldots, y_n]$ by $n-d$ linear relations in the monomials $y_i^{p_i}$.
Keeping this presentation of $R$ in mind may make reading the rest of the section easier.

One may also think of $\bX$ as an iteratited root stack, as observed in Observation 3.1.4 \cite{HIMO}. This is done as follows: $\bX_0:= \bP^d$, $\bX_i := \bX_{i-1 (\cO(\vecc),l_i,p_i)}$ with $\bX_n \cong \bX$.

\begin{theorem}\cite[Theorem 6.1.2]{HIMO}
The following object is tilting in $D^b(\bX)$: $$\cT:= \bigoplus_{\vecy \in [0,d\vecc]} \cO(\vecy).$$
\end{theorem}

Let $Q$ to be the quiver of sections of the collection of line bundles $$\{\cO(\vecy)\,|\,\vecy \in [0,d\vecc]\}\subset \Pic(\bX).$$
The algebra $\End(\cT)$ is then a quotient of $kQ$ by some ideal $I$.
We will use $\vecy$ to denote the vertex of $Q$ corresponding to $\cO(\vecy)$ and $a_{m+i}$ to denote an arrow from $\vecm$ to $\vecm+\vecy_i$ for $\vecm \in [0, d\vecc - \vecx_i]$.
Take $\bM$ to be the moduli space of quiver representations of $kQ/I$ with dimension vector $\vec{1}$ and let $S$ be its Cox ring.

As in Subsection~\ref{ssec:nice}, we have a tautological homomorphism $S \rightarrow R$.
Notice that every arrow in $Q$ is of this form $a_{m+i}$ for $\vecm \in [0, d\vecc - \vecx_i]$.
Therefore for every $m>0$ and $0\leq i\leq n$ there we have a generator $y_{a_i} - y_{a_{m+i}}$ in the corresponding ideal $I_{de}$.
Thus $S/I_{de}$ is a quotient of $k[y_0,\ldots, y_n]$: the kernel of $S/I_{de} \rightarrow k[y_{a_0},\ldots, y_{a_n}]$ is generated by the $n-d$ relations between the arrows with head at $\vec{0}$ and tail at $\vec{c}$ but those are the equations between the monomials $y_{i}^{p_i}$ in $R$.
Hence $S/I_{de} \simeq R$ and $\cT$ capture the Cox ring of $\bX$.
We thus have:

\begin{corollary} \label{cr:HIMO}
For the G-L projective space $\bX$, we have $\bX \simeq \bM^{\Pic} \simeq \bM^\theta_\textup{ref}$ for some $\textup{GIT}$ stability parameter $\theta$.
\end{corollary}

\bibliographystyle{amsplain}
\bibliography{references}

\end{document}